\documentclass[11pt]{article}

\newcommand{\yourname}{[Put Your Name Here]}
\newcommand{\yourcollaborators}{[Put Your Collaborators Here]}

\def\comments{0}

%format and packages

%\usepackage{algorithm, algorithmic}
\usepackage{algpseudocode}
\usepackage{amsmath, amssymb, amsthm}
\usepackage{enumerate}
\usepackage{enumitem}
\usepackage{framed}
\usepackage{verbatim}
\usepackage[margin=1.1in]{geometry}
\usepackage{microtype}
\usepackage{kpfonts}
\usepackage{palatino}
	\DeclareMathAlphabet{\mathtt}{OT1}{cmtt}{m}{n}
	\SetMathAlphabet{\mathtt}{bold}{OT1}{cmtt}{bx}{n}
	\DeclareMathAlphabet{\mathsf}{OT1}{cmss}{m}{n}
	\SetMathAlphabet{\mathsf}{bold}{OT1}{cmss}{bx}{n}
	\renewcommand*\ttdefault{cmtt}
	\renewcommand*\sfdefault{cmss}
	\renewcommand{\baselinestretch}{1.05}
\usepackage[usenames,dvipsnames]{xcolor}
\definecolor{DarkGreen}{rgb}{0.15,0.5,0.15}
\definecolor{DarkRed}{rgb}{0.6,0.2,0.2}
\definecolor{DarkBlue}{rgb}{0.2,0.2,0.6}
\definecolor{DarkPurple}{rgb}{0.4,0.2,0.4}
\usepackage[pdftex]{hyperref}
\hypersetup{
	linktocpage=true,
	colorlinks=true,				% false: boxed links; true: colored links
	linkcolor=DarkBlue,		% color of internal links
	citecolor=DarkBlue,	% color of links to bibliography
	urlcolor=DarkBlue,		% color of external links
}

\usepackage[boxruled,vlined,nofillcomment]{algorithm2e}
	\SetKwProg{Fn}{Function}{\string:}{}
	\SetKwFor{While}{While}{}{}
	\SetKwFor{For}{For}{}{}
	\SetKwIF{If}{ElseIf}{Else}{If}{:}{ElseIf}{Else}{:}
	\SetKw{Return}{Return}
	

%enclosure macros
\newcommand{\paren}[1]{\ensuremath{\left( {#1} \right)}}
\newcommand{\bracket}[1]{\ensuremath{\left\{ {#1} \right\}}}
\renewcommand{\sb}[1]{\ensuremath{\left[ {#1} \right\]}}
\newcommand{\ab}[1]{\ensuremath{\left\langle {#1} \right\rangle}}

%probability macros
\newcommand{\ex}[2]{{\ifx&#1& \mathbb{E} \else \underset{#1}{\mathbb{E}} \fi \left[#2\right]}}
\newcommand{\pr}[2]{{\ifx&#1& \mathbb{P} \else \underset{#1}{\mathbb{P}} \fi \left[#2\right]}}
\newcommand{\var}[2]{{\ifx&#1& \mathrm{Var} \else \underset{#1}{\mathrm{Var}} \fi \left[#2\right]}}

%useful CS macros
\newcommand{\poly}{\mathrm{poly}}
\newcommand{\polylog}{\mathrm{polylog}}
\newcommand{\zo}{\{0,1\}}
\newcommand{\pmo}{\{\pm1\}}
\newcommand{\getsr}{\gets_{\mbox{\tiny R}}}
\newcommand{\card}[1]{\left| #1 \right|}
\newcommand{\set}[1]{\left\{#1\right\}}
\newcommand{\negl}{\mathrm{negl}}
\newcommand{\eps}{\varepsilon}
\DeclareMathOperator*{\argmin}{arg\,min}
\DeclareMathOperator*{\argmax}{arg\,max}
\newcommand{\eqand}{\qquad \textrm{and} \qquad}
\newcommand{\ind}[1]{\mathbb{I}\{#1\}}
\newcommand{\sslash}{\ensuremath{\mathbin{/\mkern-3mu/}}}

%info theory macros
\newcommand{\SD}{\mathit{SD}}
\newcommand{\sd}[2]{\SD\left( #1 , #2 \right)}
\newcommand{\KL}{\mathit{KL}}
\newcommand{\kl}[2]{\KL\left(#1 \| #2 \right)}
\newcommand{\CS}{\ensuremath{\chi^2}}
\newcommand{\cs}[2]{\CS\left(#1 \| #2 \right)}
\newcommand{\MI}{\mathit{I}}
\newcommand{\mi}[2]{\MI\left(~#1~;~#2~\right)}

%mathbb
\newcommand{\N}{\mathbb{N}}
\newcommand{\R}{\mathbb{R}}
\newcommand{\Z}{\mathbb{Z}}
%mathcal
\newcommand{\cA}{\mathcal{A}}
\newcommand{\cB}{\mathcal{B}}
\newcommand{\cC}{\mathcal{C}}
\newcommand{\cD}{\mathcal{D}}
\newcommand{\cE}{\mathcal{E}}
\newcommand{\cF}{\mathcal{F}}
\newcommand{\cL}{\mathcal{L}}
\newcommand{\cM}{\mathcal{M}}
\newcommand{\cO}{\mathcal{O}}
\newcommand{\cP}{\mathcal{P}}
\newcommand{\cQ}{\mathcal{Q}}
\newcommand{\cR}{\mathcal{R}}
\newcommand{\cS}{\mathcal{S}}
\newcommand{\cU}{\mathcal{U}}
\newcommand{\cV}{\mathcal{V}}
\newcommand{\cW}{\mathcal{W}}
\newcommand{\cX}{\mathcal{X}}
\newcommand{\cY}{\mathcal{Y}}
\newcommand{\cZ}{\mathcal{Z}}

%theorem macros
\newtheorem{thm}{Theorem}
\newtheorem{lem}[thm]{Lemma}
\newtheorem{fact}[thm]{Fact}
\newtheorem{clm}[thm]{Claim}
\newtheorem{rem}[thm]{Remark}
\newtheorem{coro}[thm]{Corollary}
\newtheorem{prop}[thm]{Proposition}
\newtheorem{conj}[thm]{Conjecture}
	\theoremstyle{definition}
\newtheorem{defn}[thm]{Definition}


\newcommand{\instructor}{Paul Hand}
\newcommand{\weeknum}{2}
\newcommand{\hwnum}{1}
\newcommand{\hwdue}{Wednesday May 27, 2020 at 11:59 PM Eastern time}
%\newcommand{\hwdue}{Monday 5/4/2019 at 12:00pm via \href{https://www.gradescope.com/courses/127111}{Gradescope}}

\theoremstyle{theorem}
\newtheorem{prob}{Problem}
\newtheorem{sol}{Solution}
\newtheorem{ques}{Question}
\newtheorem{ans}{Answer}


\definecolor{cit}{rgb}{0.05,0.2,0.45} 
\newcommand{\solution}{\medskip\noindent{\color{DarkBlue}\textbf{Solution:}}}
\newcommand{\response}{\medskip\noindent{\color{DarkBlue}\textbf{Response:}}}

\begin{document}
{\Large 
\begin{center}\href{http://khoury.northeastern.edu/home/hand/teaching/cs7150-summer-2020/index.html}{CS 7150: Deep Learning} --- Summer-Full 2020 --- \instructor \end{center}}
{\large
\vspace{10pt}
\noindent HW~\hwnum\  \vspace{2pt}\\
Due: ~\hwdue \ via \href{https://www.gradescope.com/courses/127111}{Gradescope}}

\bigskip
{\large
\noindent Name: \yourname \vspace{2pt}\\ Collaborators: \yourcollaborators}

\vspace{15pt}

\noindent You may consult any and all resources.  Note that these questions are somewhat vague by design.  Part of your task is to make reasonable decisions in interpreting the questions.  Your responses should convey understanding, be written with an appropriate amount of precision, and be succinct.  Where possible, you should make precise statements.  For questions that require coding, you may either type your results with figures into this tex file, or you may append a pdf of output of a Jupyter notebook that is organized similarly.  You may use PyTorch, TensorFlow, or any other packages you like.   You may use code available on the internet as a starting point.



\begin{ques} Is a $1 \times 1$ convolution operation the same as scaling the input by a single scalar constant?  Explain.  If the answer is sometimes yes, then make sure to explain when it is and when it isn't.
\end{ques}
\response

\ 

\begin{ques} 
Show that optimization problem resulting from logistic regression is convex in the model parameters.
\end{ques}
\response

\begin{ques} 
Train a convolutional neural network for classification on the \href{https://www.cs.toronto.edu/~kriz/cifar.html}{CIFAR10} dataset.
\end{ques}

\begin{enumerate}[label=(\alph*)]
\item Clearly convey the architecture and training procedure of your network using a figure and text.

\response

\item Evaluate the performance of your network.

\response

\item Select three feature maps (channels) in three different layers of your network and visualize the content/structure within input image(s) \ that lead to large activation values of those feature maps.  You may use any algorithm you like to do this.  Clearly convey the algorithm you used.

\response

\end{enumerate}


\begin{ques} 
Perform experiments that show that BatchNorm can (a) accelerate convergence of gradient based neural network training algorithms, (b) can permit larger learning rates, and (c) can lead to better neural network performance. 
\end{ques}

\noindent Clearly explain your experimental setup, results, and interpretation.

\begin{enumerate}[label=(\alph*)]
\item \response

\item \response

\item \response
\end{enumerate}

\end{document}
